% Options for packages loaded elsewhere
\PassOptionsToPackage{unicode}{hyperref}
\PassOptionsToPackage{hyphens}{url}
%
\documentclass[
]{article}
\usepackage{amsmath,amssymb}
\usepackage{lmodern}
\usepackage{iftex}
\ifPDFTeX
  \usepackage[T1]{fontenc}
  \usepackage[utf8]{inputenc}
  \usepackage{textcomp} % provide euro and other symbols
\else % if luatex or xetex
  \usepackage{unicode-math}
  \defaultfontfeatures{Scale=MatchLowercase}
  \defaultfontfeatures[\rmfamily]{Ligatures=TeX,Scale=1}
\fi
% Use upquote if available, for straight quotes in verbatim environments
\IfFileExists{upquote.sty}{\usepackage{upquote}}{}
\IfFileExists{microtype.sty}{% use microtype if available
  \usepackage[]{microtype}
  \UseMicrotypeSet[protrusion]{basicmath} % disable protrusion for tt fonts
}{}
\makeatletter
\@ifundefined{KOMAClassName}{% if non-KOMA class
  \IfFileExists{parskip.sty}{%
    \usepackage{parskip}
  }{% else
    \setlength{\parindent}{0pt}
    \setlength{\parskip}{6pt plus 2pt minus 1pt}}
}{% if KOMA class
  \KOMAoptions{parskip=half}}
\makeatother
\usepackage{xcolor}
\IfFileExists{xurl.sty}{\usepackage{xurl}}{} % add URL line breaks if available
\IfFileExists{bookmark.sty}{\usepackage{bookmark}}{\usepackage{hyperref}}
\hypersetup{
  pdftitle={R Notebook},
  hidelinks,
  pdfcreator={LaTeX via pandoc}}
\urlstyle{same} % disable monospaced font for URLs
\usepackage[margin=1in]{geometry}
\usepackage{color}
\usepackage{fancyvrb}
\newcommand{\VerbBar}{|}
\newcommand{\VERB}{\Verb[commandchars=\\\{\}]}
\DefineVerbatimEnvironment{Highlighting}{Verbatim}{commandchars=\\\{\}}
% Add ',fontsize=\small' for more characters per line
\usepackage{framed}
\definecolor{shadecolor}{RGB}{248,248,248}
\newenvironment{Shaded}{\begin{snugshade}}{\end{snugshade}}
\newcommand{\AlertTok}[1]{\textcolor[rgb]{0.94,0.16,0.16}{#1}}
\newcommand{\AnnotationTok}[1]{\textcolor[rgb]{0.56,0.35,0.01}{\textbf{\textit{#1}}}}
\newcommand{\AttributeTok}[1]{\textcolor[rgb]{0.77,0.63,0.00}{#1}}
\newcommand{\BaseNTok}[1]{\textcolor[rgb]{0.00,0.00,0.81}{#1}}
\newcommand{\BuiltInTok}[1]{#1}
\newcommand{\CharTok}[1]{\textcolor[rgb]{0.31,0.60,0.02}{#1}}
\newcommand{\CommentTok}[1]{\textcolor[rgb]{0.56,0.35,0.01}{\textit{#1}}}
\newcommand{\CommentVarTok}[1]{\textcolor[rgb]{0.56,0.35,0.01}{\textbf{\textit{#1}}}}
\newcommand{\ConstantTok}[1]{\textcolor[rgb]{0.00,0.00,0.00}{#1}}
\newcommand{\ControlFlowTok}[1]{\textcolor[rgb]{0.13,0.29,0.53}{\textbf{#1}}}
\newcommand{\DataTypeTok}[1]{\textcolor[rgb]{0.13,0.29,0.53}{#1}}
\newcommand{\DecValTok}[1]{\textcolor[rgb]{0.00,0.00,0.81}{#1}}
\newcommand{\DocumentationTok}[1]{\textcolor[rgb]{0.56,0.35,0.01}{\textbf{\textit{#1}}}}
\newcommand{\ErrorTok}[1]{\textcolor[rgb]{0.64,0.00,0.00}{\textbf{#1}}}
\newcommand{\ExtensionTok}[1]{#1}
\newcommand{\FloatTok}[1]{\textcolor[rgb]{0.00,0.00,0.81}{#1}}
\newcommand{\FunctionTok}[1]{\textcolor[rgb]{0.00,0.00,0.00}{#1}}
\newcommand{\ImportTok}[1]{#1}
\newcommand{\InformationTok}[1]{\textcolor[rgb]{0.56,0.35,0.01}{\textbf{\textit{#1}}}}
\newcommand{\KeywordTok}[1]{\textcolor[rgb]{0.13,0.29,0.53}{\textbf{#1}}}
\newcommand{\NormalTok}[1]{#1}
\newcommand{\OperatorTok}[1]{\textcolor[rgb]{0.81,0.36,0.00}{\textbf{#1}}}
\newcommand{\OtherTok}[1]{\textcolor[rgb]{0.56,0.35,0.01}{#1}}
\newcommand{\PreprocessorTok}[1]{\textcolor[rgb]{0.56,0.35,0.01}{\textit{#1}}}
\newcommand{\RegionMarkerTok}[1]{#1}
\newcommand{\SpecialCharTok}[1]{\textcolor[rgb]{0.00,0.00,0.00}{#1}}
\newcommand{\SpecialStringTok}[1]{\textcolor[rgb]{0.31,0.60,0.02}{#1}}
\newcommand{\StringTok}[1]{\textcolor[rgb]{0.31,0.60,0.02}{#1}}
\newcommand{\VariableTok}[1]{\textcolor[rgb]{0.00,0.00,0.00}{#1}}
\newcommand{\VerbatimStringTok}[1]{\textcolor[rgb]{0.31,0.60,0.02}{#1}}
\newcommand{\WarningTok}[1]{\textcolor[rgb]{0.56,0.35,0.01}{\textbf{\textit{#1}}}}
\usepackage{graphicx}
\makeatletter
\def\maxwidth{\ifdim\Gin@nat@width>\linewidth\linewidth\else\Gin@nat@width\fi}
\def\maxheight{\ifdim\Gin@nat@height>\textheight\textheight\else\Gin@nat@height\fi}
\makeatother
% Scale images if necessary, so that they will not overflow the page
% margins by default, and it is still possible to overwrite the defaults
% using explicit options in \includegraphics[width, height, ...]{}
\setkeys{Gin}{width=\maxwidth,height=\maxheight,keepaspectratio}
% Set default figure placement to htbp
\makeatletter
\def\fps@figure{htbp}
\makeatother
\setlength{\emergencystretch}{3em} % prevent overfull lines
\providecommand{\tightlist}{%
  \setlength{\itemsep}{0pt}\setlength{\parskip}{0pt}}
\setcounter{secnumdepth}{-\maxdimen} % remove section numbering
\ifLuaTeX
  \usepackage{selnolig}  % disable illegal ligatures
\fi

\title{R Notebook}
\author{}
\date{\vspace{-2.5em}}

\begin{document}
\maketitle

This is an \href{http://rmarkdown.rstudio.com}{R Markdown} Notebook.
When you execute code within the notebook, the results appear beneath
the code.

Try executing this chunk by clicking the \emph{Run} button within the
chunk or by placing your cursor inside it and pressing
\emph{Ctrl+Shift+Enter}.

\#Task 1

\begin{Shaded}
\begin{Highlighting}[]
\FunctionTok{library}\NormalTok{(dplyr)}
\end{Highlighting}
\end{Shaded}

\begin{verbatim}
## 
## Attaching package: 'dplyr'
\end{verbatim}

\begin{verbatim}
## The following objects are masked from 'package:stats':
## 
##     filter, lag
\end{verbatim}

\begin{verbatim}
## The following objects are masked from 'package:base':
## 
##     intersect, setdiff, setequal, union
\end{verbatim}

\begin{Shaded}
\begin{Highlighting}[]
\FunctionTok{library}\NormalTok{(tidyverse)}
\end{Highlighting}
\end{Shaded}

\begin{verbatim}
## -- Attaching packages --------------------------------------- tidyverse 1.3.0 --
\end{verbatim}

\begin{verbatim}
## v ggplot2 3.3.3     v purrr   0.3.4
## v tibble  3.1.0     v stringr 1.4.0
## v tidyr   1.1.3     v forcats 0.5.1
## v readr   1.4.0
\end{verbatim}

\begin{verbatim}
## -- Conflicts ------------------------------------------ tidyverse_conflicts() --
## x dplyr::filter() masks stats::filter()
## x dplyr::lag()    masks stats::lag()
\end{verbatim}

\begin{Shaded}
\begin{Highlighting}[]
\NormalTok{sample1 }\OtherTok{\textless{}{-}} \FunctionTok{read.csv}\NormalTok{(}\AttributeTok{file =} \StringTok{\textquotesingle{}Employee Report Sample Data 1.csv\textquotesingle{}}\NormalTok{)}
\NormalTok{sample2 }\OtherTok{\textless{}{-}} \FunctionTok{read.csv}\NormalTok{(}\AttributeTok{file =} \StringTok{\textquotesingle{}Employee Report Sample Data 2.csv\textquotesingle{}}\NormalTok{)}
\NormalTok{rep\_to }\OtherTok{\textless{}{-}}\NormalTok{ sample1 }\SpecialCharTok{\%\textgreater{}\%}
  \FunctionTok{group\_by}\NormalTok{(ReportsTo) }\SpecialCharTok{\%\textgreater{}\%}
  \FunctionTok{summarize}\NormalTok{(}\FunctionTok{n}\NormalTok{(), }\FunctionTok{as.integer}\NormalTok{(}\FunctionTok{mean}\NormalTok{(Age)))}

\FunctionTok{names}\NormalTok{(rep\_to) }\OtherTok{\textless{}{-}} \FunctionTok{c}\NormalTok{(}\StringTok{\textquotesingle{}Team Member Reported To\textquotesingle{}}\NormalTok{, }\StringTok{\textquotesingle{}Number of Team Members\textquotesingle{}}\NormalTok{, }\StringTok{\textquotesingle{}Average Age\textquotesingle{}}\NormalTok{)}
\NormalTok{task1 }\OtherTok{\textless{}{-}}\NormalTok{ rep\_to}
\NormalTok{task1}
\end{Highlighting}
\end{Shaded}

\begin{verbatim}
## # A tibble: 9 x 3
##   `Team Member Reported To` `Number of Team Members` `Average Age`
##   <chr>                                        <int>         <int>
## 1 ""                                               1            45
## 2 "Andre Perez"                                    5            25
## 3 "Andrew Smith"                                   2            22
## 4 "Bob Boss"                                       4            29
## 5 "Daniel Smith"                                   2            28
## 6 "David S"                                        2            26
## 7 "Jenny Richards"                                 6            41
## 8 "Katie Walker"                                   2            30
## 9 "Martha Jones"                                   1            29
\end{verbatim}

\#Task 2

\begin{Shaded}
\begin{Highlighting}[]
\FunctionTok{names}\NormalTok{(sample2) }\OtherTok{\textless{}{-}} \FunctionTok{c}\NormalTok{(}\StringTok{\textquotesingle{}FullName\textquotesingle{}}\NormalTok{, }\StringTok{\textquotesingle{}StartDate\textquotesingle{}}\NormalTok{)}

\NormalTok{sample2}\SpecialCharTok{$}\NormalTok{StartDate }\OtherTok{\textless{}{-}} \FunctionTok{as.Date}\NormalTok{(sample2}\SpecialCharTok{$}\NormalTok{StartDate, }\AttributeTok{format =} \StringTok{\textquotesingle{}\%m/\%d/\%y\textquotesingle{}}\NormalTok{ )}
\NormalTok{sample2}\SpecialCharTok{$}\NormalTok{Tenure }\OtherTok{\textless{}{-}} \FunctionTok{round}\NormalTok{(}\FunctionTok{as.numeric}\NormalTok{((}\FunctionTok{Sys.Date}\NormalTok{() }\SpecialCharTok{{-}}\NormalTok{ sample2}\SpecialCharTok{$}\NormalTok{StartDate)}\SpecialCharTok{/}\DecValTok{365}\NormalTok{),}\DecValTok{1}\NormalTok{)}
\NormalTok{subset2 }\OtherTok{\textless{}{-}}\NormalTok{ sample2 }\SpecialCharTok{\%\textgreater{}\%}
  \FunctionTok{select}\NormalTok{(FullName, Tenure)}

\NormalTok{sample1}\SpecialCharTok{$}\NormalTok{FullName }\OtherTok{\textless{}{-}} \FunctionTok{paste}\NormalTok{(sample1}\SpecialCharTok{$}\NormalTok{FirstName, sample1}\SpecialCharTok{$}\NormalTok{LastName, }\AttributeTok{sep=}\StringTok{\textquotesingle{} \textquotesingle{}}\NormalTok{)}
\NormalTok{subset1 }\OtherTok{\textless{}{-}}\NormalTok{ sample1 }\SpecialCharTok{\%\textgreater{}\%}
  \FunctionTok{select}\NormalTok{(FullName, Position)}

\NormalTok{pos\_ten }\OtherTok{\textless{}{-}} \FunctionTok{merge}\NormalTok{(subset1, subset2, }\AttributeTok{by=}\StringTok{"FullName"}\NormalTok{)}
\NormalTok{task2 }\OtherTok{\textless{}{-}}\NormalTok{ pos\_ten }\SpecialCharTok{\%\textgreater{}\%}
  \FunctionTok{select}\NormalTok{(Position, Tenure)}
\NormalTok{task2[}\FunctionTok{order}\NormalTok{(}\SpecialCharTok{{-}}\NormalTok{task2}\SpecialCharTok{$}\NormalTok{Tenure),]}
\end{Highlighting}
\end{Shaded}

\begin{verbatim}
##           Position Tenure
## 16         Analyst    9.7
## 14             CFO    9.2
## 15       Assistant    9.0
## 17      Contractor    7.4
## 20         Analyst    7.3
## 4           Intern    7.0
## 13          Intern    6.8
## 25         Analyst    6.8
## 8         Engineer    6.0
## 18 Finance Manager    5.9
## 10        Director    5.3
## 2         Director    4.5
## 21       Assistant    4.5
## 3          Analyst    3.9
## 12             CEO    3.8
## 24        Engineer    3.8
## 22       Assistant    3.7
## 9         Engineer    3.3
## 1       Accountant    3.2
## 11         Analyst    2.7
## 23           Sales    2.6
## 5        Assistant    1.8
## 6              CTO    0.5
## 7               HR   -0.1
## 19           Sales   -0.2
\end{verbatim}

\#Task 3

\begin{Shaded}
\begin{Highlighting}[]
\FunctionTok{names}\NormalTok{(sample1) }\OtherTok{\textless{}{-}} \FunctionTok{c}\NormalTok{(}\StringTok{\textquotesingle{}ID\textquotesingle{}}\NormalTok{, }\StringTok{\textquotesingle{}FirstName\textquotesingle{}}\NormalTok{, }\StringTok{\textquotesingle{}LastName\textquotesingle{}}\NormalTok{, }\StringTok{\textquotesingle{}ReportsTo\textquotesingle{}}\NormalTok{, }\StringTok{\textquotesingle{}Position\textquotesingle{}}\NormalTok{, }\StringTok{\textquotesingle{}Age\textquotesingle{}}\NormalTok{, }\StringTok{\textquotesingle{}FullName\textquotesingle{}}\NormalTok{)}
\NormalTok{task3 }\OtherTok{\textless{}{-}}\NormalTok{ sample1}\SpecialCharTok{$}\NormalTok{ID[}\FunctionTok{substr}\NormalTok{(sample1}\SpecialCharTok{$}\NormalTok{FirstName, }\DecValTok{1}\NormalTok{, }\DecValTok{1}\NormalTok{) }\SpecialCharTok{==} \StringTok{\textquotesingle{}M\textquotesingle{}}\NormalTok{]}
\NormalTok{task3}
\end{Highlighting}
\end{Shaded}

\begin{verbatim}
## [1]  2 12 20 23 25
\end{verbatim}

\#Task 4

\begin{Shaded}
\begin{Highlighting}[]
\FunctionTok{library}\NormalTok{(ggplot2)}
\FunctionTok{library}\NormalTok{(plotly)}
\end{Highlighting}
\end{Shaded}

\begin{verbatim}
## 
## Attaching package: 'plotly'
\end{verbatim}

\begin{verbatim}
## The following object is masked from 'package:ggplot2':
## 
##     last_plot
\end{verbatim}

\begin{verbatim}
## The following object is masked from 'package:stats':
## 
##     filter
\end{verbatim}

\begin{verbatim}
## The following object is masked from 'package:graphics':
## 
##     layout
\end{verbatim}

\begin{Shaded}
\begin{Highlighting}[]
\NormalTok{subset3 }\OtherTok{\textless{}{-}}\NormalTok{ sample1 }\SpecialCharTok{\%\textgreater{}\%}
  \FunctionTok{select}\NormalTok{(FullName, Age, Position)}

\NormalTok{age\_ten }\OtherTok{\textless{}{-}} \FunctionTok{merge}\NormalTok{(subset3, subset2, }\AttributeTok{by =} \StringTok{"FullName"}\NormalTok{)}

\FunctionTok{ggplot}\NormalTok{(age\_ten, }\FunctionTok{aes}\NormalTok{(}\AttributeTok{x=}\NormalTok{Age, }\AttributeTok{y=}\NormalTok{Tenure, }\AttributeTok{color=}\NormalTok{Position)) }\SpecialCharTok{+} \FunctionTok{geom\_point}\NormalTok{()}
\end{Highlighting}
\end{Shaded}

\includegraphics{Sample_Task_files/figure-latex/unnamed-chunk-4-1.pdf}

Add a new chunk by clicking the \emph{Insert Chunk} button on the
toolbar or by pressing \emph{Ctrl+Alt+I}.

When you save the notebook, an HTML file containing the code and output
will be saved alongside it (click the \emph{Preview} button or press
\emph{Ctrl+Shift+K} to preview the HTML file).

The preview shows you a rendered HTML copy of the contents of the
editor. Consequently, unlike \emph{Knit}, \emph{Preview} does not run
any R code chunks. Instead, the output of the chunk when it was last run
in the editor is displayed.

\end{document}
